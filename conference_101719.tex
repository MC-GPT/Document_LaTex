\documentclass[conference]{IEEEtran}
\IEEEoverridecommandlockouts
% The preceding line is only needed to identify funding in the first footnote. If that is unneeded, please comment it out.
\usepackage{cite}
\usepackage{amsmath,amssymb,amsfonts}
\usepackage{algorithmic}
\usepackage{graphicx}
\usepackage{textcomp}
\usepackage{longtable}
\usepackage{xcolor}
\def\BibTeX{{\rm B\kern-.05em{\sc i\kern-.025em b}\kern-.08em
    T\kern-.1667em\lower.7ex\hbox{E}\kern-.125emX}}
\begin{document}

\title{MC NUGU\\
{\footnotesize \textsuperscript{*}- AI speakers can be the "MC of the party" -}
}

\author{\IEEEauthorblockN{1\textsuperscript{st} Kang Jungyu}
\IEEEauthorblockA{\textit{Dept. of Information Systems } \\
\textit{Hanyang Univ.}\\
Seoul, Republic of Korea \\
jgpk226@gmail.com}
\and
\IEEEauthorblockN{2\textsuperscript{nd} Kim Jihoon}
\IEEEauthorblockA{\textit{Dept. of Information Systems } \\
\textit{Hanyang Univ.}\\
Seoul, Republic of Korea \\
jhkimlego@gmail.com}
\and
\IEEEauthorblockN{3\textsuperscript{rd} Park Jongsu}
\IEEEauthorblockA{\textit{Dept. of Information Systems } \\
\textit{Hanyang Univ.}\\
Seoul, Republic of Korea \\
qwt629@gmail.com}
}

\maketitle

\begin{abstract}
The MC NUGU - AI speakers can be the "MC of the party” is Icebreaker for MT, meetings, and house parties. It can also act as a party assistant in place of a human.

In addition, it provides a service that connects LG home appliances with the app and allows you to control them from the app. Key features include lightning services, the ability to adjust and control the brightness of colors, change house party mode, and get lighting recommendations through generative AI.

\end{abstract}

\begin{IEEEkeywords}
MC, Manager, Lighting, Appliance, Game
\end{IEEEkeywords}

 \begin{tabular}{|p{1.2cm}|p{1.5cm}|p{5cm}|} 
 \hline
 Name & Roles & Task description and etc. \\ [0.5ex] 
 \hline
 Kang Jungyu & Development Manager, Software developer (Back-end) &  
 A Development Manager is responsible for managing the overall project. He oversees the functions that are required, the UI/UX configuration, the flawless operation of the prototype, and whether the project proceeds as planned. In addition, he constantly monitors the team member’s progress and whether they are communicating effectively.
 
 Software developer (Back-end) is in charge of creating databases, web services, and APIs used by front-end developers. For this position, an understanding of the core database and its features is essential. Every logic as well as integrating with the front-end is developed by him. He uses two tracks of servers, which are the main server and the AI server. The main server interacts with the database, manages user authentication, and handles most server-side logic by Java Spring. The AI Server uses AI codes, open AI, and other libraries for games by Python Flask. He also writes HTTP request codes and WebSocket connection codes on Front-end using React-Native for this project.\\ 
 \hline
 Kim Jihoon & Product Designer, Software developer (AI) & 
 A Product Designer
 A software developer (AI) is responsible for any requirements of the project in terms of data and artificial intelligence. This role selects and refine data into adequate forms in order to make the artificial intelligence model. This role uses machine learning for implementing artificial intelligence to requirements of the software, must run and test model and make a statistical analysis.\\
 \hline
 Park Jongsu & Software developer (Front-end) & 
 A software developer is a type of software developer who specializes in creating and designing the user interface (UI) and user experience (UX) of websites and web applications. The primary responsibility of a front-end developer is to ensure that the visual and interactive aspects of a website or application are user-friendly, aesthetically pleasing, and functionally efficient.
 
In this project, he creates  the client side of  a mobile app with Javascript using React-Native framework, and git.  Anything that users see and click is made by him. He determines the components, navigations, contexts, screens and layouts.  He communicates with Designer to improve appearance and Back-end Developer to ensure running applications smoothly.
\\ [1ex] 
 \hline
\end{tabular}

\section{Introduction}

\subsection{Motivation and Term Definition}
\subsection{Problem Statements}
\subsection{Solution}
\subsection{Research on Any Relative Software}

\section{Requirement Analysis}

\subsection{Logging In}
\subsection{Creating Account}
\subsection{Space Management}
\subsection{Status Dashboard}
\subsection{Lighting Control}
\subsection{Appliance Control}
\subsection{Party Games}

\section{Development environment}

\subsection{Choice of Software}
\subsection{Cost Estimation}

\subsection{Software in Use}
\begin{figure}[htbp]
\centerline{\includegraphics[width=2.5cm]{ReactNative.png}}
\label{fig}
\caption{Logo of GitHub}
\end{figure}
\subsubsection{GitHub}

\section{Specification}

\subsection{Loading Page}
\subsection{Login Page}
\subsection{Register Page}
\subsection{Space List Page}
\subsection{Space Registration Pop-Up}
\subsection{Space Entrance Pop-Up}
\subsection{Dashboard (Main Page)}
\subsection{Lighting Dashboard}
\subsection{Lighting Device Registration Pop-Up}
\subsection{Lighting Control Page}
\subsection{Game Dashboard}
\subsection{Game Host Page}
\subsection{Game Participant Page}
\subsection{AI Guessing Game}
\subsection{Geography Guessing Game}
\subsection{'Gather Up' Game}
\subsection{App Setting Page}

\section{ARCHITECTURE DESIGN}

\section{use cases}

\section{conclusion \& discussion}

\section*{Acknowledgment}

\section*{References}

Please number citations consecutively within brackets \cite{b1}. The 
sentence punctuation follows the bracket \cite{b2}. Refer simply to the reference 
number, as in \cite{b3}---do not use ``Ref. \cite{b3}'' or ``reference \cite{b3}'' except at 
the beginning of a sentence: ``Reference \cite{b3} was the first $\ldots$''

Number footnotes separately in superscripts. Place the actual footnote at 
the bottom of the column in which it was cited. Do not put footnotes in the 
abstract or reference list. Use letters for table footnotes.

Unless there are six authors or more give all authors' names; do not use 
``et al.''. Papers that have not been published, even if they have been 
submitted for publication, should be cited as ``unpublished'' \cite{b4}. Papers 
that have been accepted for publication should be cited as ``in press'' \cite{b5}. 
Capitalize only the first word in a paper title, except for proper nouns and 
element symbols.

For papers published in translation journals, please give the English 
citation first, followed by the original foreign-language citation \cite{b6}.

\begin{thebibliography}{00}
\bibitem{b1} G. Eason, B. Noble, and I. N. Sneddon, ``On certain integrals of Lipschitz-Hankel type involving products of Bessel functions,'' Phil. Trans. Roy. Soc. London, vol. A247, pp. 529--551, April 1955.
\bibitem{b2} J. Clerk Maxwell, A Treatise on Electricity and Magnetism, 3rd ed., vol. 2. Oxford: Clarendon, 1892, pp.68--73.
\bibitem{b3} I. S. Jacobs and C. P. Bean, ``Fine particles, thin films and exchange anisotropy,'' in Magnetism, vol. III, G. T. Rado and H. Suhl, Eds. New York: Academic, 1963, pp. 271--350.
\bibitem{b4} K. Elissa, ``Title of paper if known,'' unpublished.
\bibitem{b5} R. Nicole, ``Title of paper with only first word capitalized,'' J. Name Stand. Abbrev., in press.
\bibitem{b6} Y. Yorozu, M. Hirano, K. Oka, and Y. Tagawa, ``Electron spectroscopy studies on magneto-optical media and plastic substrate interface,'' IEEE Transl. J. Magn. Japan, vol. 2, pp. 740--741, August 1987 [Digests 9th Annual Conf. Magnetics Japan, p. 301, 1982].
\bibitem{b7} M. Young, The Technical Writer's Handbook. Mill Valley, CA: University Science, 1989.
\end{thebibliography}
\vspace{12pt}
\color{red}
IEEE conference templates contain guidance text for composing and formatting conference papers. Please ensure that all template text is removed from your conference paper prior to submission to the conference. Failure to remove the template text from your paper may result in your paper not being published.

\end{document}
